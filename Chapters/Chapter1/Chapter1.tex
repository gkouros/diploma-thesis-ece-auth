\chapter{Εισαγωγή} \label{Chapter1}
Η Ρομποτική αποτελεί έναν σχετικά μικρό σε ηλικία επιστημονικό πεδίο που βασίζεται σε μία πληθώρα διαφορετικών κλάδων, όπως η μηχανική, η ηλεκτρολογία, η ηλεκτρονική, η επιστήμη των υπολογιστών, ενώ, ακόμα και σε κοινωνικές επιστήμες και ασχολείται με την σχεδίαση, υλοποίηση και λειτουργία ρομποτικών συστημάτων για την αυτοματοποίηση ενός ευρέος συνόλου εργασιών.

\bigskip
Το κύριο πεδίο εφαρμογής της ρομποτικής, αποτελούσε μέχρι πρόσφατα, ο τομέας της βιομηχανίας, μέσω της εκμετάλλευσης ρομποτικών βραχιόνων για την αυτοματοποίηση των αλυσίδων παραγωγής προϊόντων σε εργοστάσια. Τα τελευταία χρόνια, παρόλα αυτά, έχει επέλθει μεγάλη ανάπτυξη στον κλάδο της ρομποτικής με την εμφάνιση εναλλακτικών πεδίων εφαρμογής ρομποτικών συστημάτων, με ιδιαίτερη έμφαση στα αυτόνομα και ημιαυτόνομα ρομπότ που έχουν σκοπό την παροχή υπηρεσιών για την βελτίωση της ποιότητας της ζωής και της ασφάλειας των ανθρώπων. Σ' αυτή την κατηγορία μπορεί να ανήκουν ρομπότ καθαρισμού, επίβλεψης, συντήρησης, διασωστικά ρομπότ, ρομπότ υπηρέτες, ενώ εξαιρετικά μεγάλο ενδιαφέρον παρουσιάζει ο ραγδαία εξελισσόμενος, τα τελευταία χρόνια, κλάδος των αυτόνομων αυτοκινήτων.

\bigskip
Ως αυτόνομο ρομπότ (Autonomous Mobile Robot ή AMR) ορίζεται ένα ρομποτικό σύστημα, το οποίο μπορεί να πλοηγηθεί αυτόνομα σε ένα γνωστό ή ακόμα και άγνωστο περιβάλλον, ενώ πραγματοποιεί ταυτόχρονα διάφορες εργασίες για την επίτευξη ενός δεδομένου στόχου. Τα αυτόνομα ρομπότ διακρίνονται σε κατηγορίες, ανάλογα με τον χώρο εργασίας τους, σε αυτόνομα οχήματα εδάφους (Autonomous Ground Vehicles ή AGVs), νερού (Autonomous Water/Underwater Vehicles ή AWV/AUV) και αέρα (Autonomous Aerial Vehicles ή AAVs), ενώ παράλληλα υπάρχουν και αυτόνομα ανθρωποειδή ρομπότ (Humanoid Robots). Στην παρούσα διπλωματική εργασία εστιαζόμαστε στα αυτόνομα ρομποτικά οχήματα εδάφους, με έμφαση στα ρομποτικά οχήματα που λειτουργούν όμοια με το συμβατικό αυτοκίνητο (Car-like Robots).

\bigskip
Για την επίτευξη της αυτόνομης λειτουργίας, ένα αυτόνομο ρομπότ, θα πρέπει να μπορεί να κινείται, να αντιλαμβάνεται το περιβάλλον του και να προσαρμόζεται σε αυτό. Για την πραγματοποίηση μία εργασίας, θα πρέπει να μπορεί να λαμβάνει τις απαραίτητες πληροφορίες από το περιβάλλον του, να τις επεξεργάζεται κατάλληλα και να αλληλεπιδρά με αυτό, ενώ παράλληλα, θα πρέπει να μπορεί να λειτουργεί σε πραγματικό χρόνο και χωρίς να θέτει τον εαυτό του ή το περιβάλλον του σε κίνδυνο.

%----------------------------------------------------------------------------------------
%	SECTION 1
%----------------------------------------------------------------------------------------

\section{Περιγραφή του Προβλήματος}
Η παρούσα διπλωματική εργασία εξετάζει το πρόβλημα της αυτόνομης πλοήγησης σε άγνωστο περιβάλλον, ενός αυτόνομου ρομποτικού οχήματος εδάφους που ανήκει στην κατηγορία των Car-Like Robots, δηλαδή, του οποίου η κίνηση στο επίπεδο βασίζεται στην κίνηση των συμβατικών αυτοκινήτων και άρα, το οποίο επιβαρύνεται με αντίστοιχους περιορισμούς. Συγκεκριμένα, εξετάζεται η περίπτωση ενός αυτόνομου ρομποτικού οχήματος εδάφους με κινηματικό μοντέλο Τετραδιεύθυνσης (4-Wheel-Steering), που αποτελεί μία επέκταση του συμβατικού αυτοκινήτου με δυνατότητα ταυτόχρονης στρέψης των μπροστινών και των πίσω τροχών, πετυχαίνοντας έτσι μεγαλύτερη ευελιξία και μεγαλύτερο εύρος δυνατών κινήσεων.

\bigskip
Το πρόβλημα της αυτόνομης πλοήγησης ενός ρομποτικού οχήματος εδάφους, αποτελεί ένα εξαιρετικά μελετημένο πρόβλημα, στον τομέα της ρομποτικής. Παρόλα αυτά, στις περισσότερες περιπτώσεις, οι ερευνητές που μελετούν το πρόβλημα, εστιάζονται σε ρομποτικά οχήματα πολύ περισσότερο ευέλικτα από το συμβατικό αυτοκίνητο, όπως για παράδειγμα τα Πανκατευθυντικά (Omnidirectional) ρομπότ που μπορούν να κινηθούν προς οποιαδήποτε κατεύθυνση στο επίπεδο, με δυνατότητα ταυτόχρονης περιστροφής γύρω από τον άξονα τους, αλλά και τα Διαφορικά (Differential) ρομπότ ή ρομπότ με Skid-Steering, που διαθέτουν δυνατότητα επιτόπου περιστροφής (0-point-turning). Σε αντίθεση με τα παραπάνω είδη, τα ρομποτικά οχήματα που εξετάζονται, είναι λιγότερο ευέλικτα λόγω κινηματικών περιορισμών που έχουν σαν αποτέλεσμα, τον περιορισμό των δυνατών κινήσεων, μέσω της φραγμένης ακτίνας της τροχιάς που διαγράφουν. Αυτό το γεγονός, περιπλέκει σημαντικά το πρόβλημα της αυτόνομης πλοήγησης, κάτι το οποίο, γίνεται φανερό, αν αναλογιστεί κανείς, ότι για να μπορέσει ένα τέτοιο ρομπότ να αντιστρέψει τον προσανατολισμό του θα πρέπει να εκτελέσει μία ακολουθία ελιγμών.

\bigskip
Το πρόβλημα της αυτόνομης πλοήγησης, ουσιαστικά, έγκειται στην εύρεση μίας μεθόδου που θα επιτρέψει την μετάβαση του ρομπότ από μία αρχική θέση σε μία τελική. Για να μπορεί να επιτευχθεί, όμως, ο εν λόγω στόχος, το ρομπότ θα πρέπει, ανά πάσα στιγμή, να γνωρίζει που βρίσκεται, αλλά και που πρέπει να πάει, είτε σε γνωστό, είτε σε άγνωστο περιβάλλον. Επομένως, παράλληλα με το πρόβλημα της αυτόνομης πλοήγησης εξετάζεται και το πρόβλημα της χαρτογράφησης ενός άγνωστου περιβάλλοντος, όπως επίσης και το πρόβλημα του εντοπισμού θέσης του ρομπότ.

\bigskip
Η εκπόνηση της παρούσας διπλωματικής εργασίας πραγματοποιήθηκε στα πλαίσια της ομάδας ρομποτικής P.A.N.D.O.R.A, η οποία δραστηριοποιείται στην ανάπτυξη αυτόνομων ρομποτικών οχημάτων, με τα οποία συμμετέχει σε διεθνείς διαγωνισμούς ρομποτικής, όπως για παράδειγμα ο διαγωνισμός RoboCup Rescue, στον οποίο έχει λάβει δύο φορές την δεύτερη θέση στην κατηγορία αυτόνομων διασωστικών ρομπότ. Αφορμή για την εκπόνηση της παρούσας εργασίας αποτέλεσε η επιθυμία για πειραματισμό με εναλλακτικά μοντέλα κίνησης οχημάτων, πέρα από τα συνήθη Differential και Skid-Steering, αλλά και η επιθυμία για ανάπτυξη ενός δεύτερου ρομποτικού οχήματος για μεταγενέστερο πειραματισμό με συνεργατική λειτουργία πολλαπλών ρομποτικών πρακτόρων.

%----------------------------------------------------------------------------------------
%	SECTION 2
%----------------------------------------------------------------------------------------

\section{Συνεισφορά της Διπλωματικής}
Στα πλαίσια εκπόνησης της παρούσας διπλωματικής εργασίας αναπτύχθηκε η αυτόνομη ρομποτική πλατφόρμα Monstertruck, η οποία αποτελεί ένα ρομποτικό όχημα που περιγράφεται από ένα μη ιδανικό κινηματικό μοντέλο Τετραδιεύθυνσης. Η ρομποτική πλατφόρμα Monstertruck χρησιμοποιεί ένα σύνολο αισθητήρων και κινητήρων που της επιτρέπουν να αντιλαμβάνεται το περιβάλλον και να κινείται μέσα σε αυτό, αυτόνομα.

\bigskip
Για να καθίσταται δυνατή η επίτευξη της αυτόνομης συμπεριφοράς της ρομποτικής πλατφόρμας Monstertruck εξετάστηκαν δύο διαφορετικά συστήματα χαρτογράφησης και εντοπισμού θέσης που ακολουθούν διαφορετικές προσεγγίσεις και κάθε μία παρουσιάζει ένα σύνολο πλεονεκτημάτων, αλλά και μειονεκτημάτων έναντι της άλλης, ενώ και τα δύο ταυτόχρονα προορίζονται για χρήση σε εσωτερικούς και δομημένους χώρους.

\bigskip
Βασική συνεισφορά της παρούσας διπλωματικής εργασίας αποτελούν δύο συστήματα αυτόνομης πλοήγησης που υλοποιήθηκαν με στόχο την επίτευξη εφικτής, ασφαλούς, ομαλής και επιτυχούς αυτόνομης πλοήγησης της ρομποτικής πλατφόρμας Monstertruck και γενικότερα οποιουδήποτε ρομποτικού οχήματος με τετραδιεύθυνση. Η βασική διαφορά μεταξύ των δύο συστημάτων, έγκειται στο στάδιο λειτουργίας της δυναμικής συμπεριφοράς τους, με το ένα να συνδυάζει στατικό σχεδιασμό και δυναμική εκτέλεση, ενώ το δεύτερο να συνδυάζει δυναμικό σχεδιασμό και στατική εκτέλεση.

%----------------------------------------------------------------------------------------
%	SECTION 3
%----------------------------------------------------------------------------------------

\section{Διάρθρωση της Διπλωματικής}
Στα επόμενα κεφάλαια θα παρουσιαστεί αναλυτικά η ανάπτυξη της ρομποτικής πλατφόρμας Monstertruck και των αντίστοιχων υποσυστημάτων αυτής που καθιστούν δυνατή την αυτόνομη λειτουργία της. Συγκεκριμένα:

\begin{itemize}
	\item \textbf{Στο Κεφάλαιο~\ref{Chapter2}} παρουσιάζεται η ρομποτική πλατφόρμα Monstertruck, όσον αφορά την μηχανολογική κατασκευή και τον ηλεκτρολογικό και ηλεκτρονικό εξοπλισμό που διαθέτει, με έμφαση στους αισθητήρες και στους κινητήρες. Επιπλέον, παρουσιάζεται η κινηματική ανάλυση της ρομποτικής πλατφόρμας Monstertruck, όσον αφορά τη μετάδοση κίνησης των κινητήρων στους τροχούς, όπως επίσης και τη σχέση μεταξύ της κίνησης των τροχών με την κίνηση του οχήματος στο επίπεδο.
	\item Στο \textbf{Κεφάλαιο~\ref{Chapter3}} παρουσιάζονται οι αλγόριθμοι και οι μέθοδοι που χρησιμοποιήθηκαν για την επίτευξη της αυτόνομης συμπεριφοράς της ρομποτικής πλατφόρμας Monstertruck, όσον αφορά την χαρτογράφηση και τον εντοπισμό θέσης, αλλά και τον σχεδιασμό και την εκτέλεση της αυτόνομης πλοήγησης στο επίπεδο.
	\item Στο \textbf{Κεφάλαιο~\ref{Chapter4}} παρουσιάζεται η αρχιτεκτονική του συστήματος λογισμικού που υλοποιήθηκε για την ρομποτική πλατφόρμα Monstertruck, όπως επίσης και τα εργαλεία που χρησιμοποιήθηκαν, με έμφαση στο meta-λειτουργικό σύστημα ROS (Robotic Operating System) και τα εργαλεία προσομοίωσης STDR και Gazebo.
	\item Στο \textbf{Κεφάλαιο~\ref{Chapter5}} παρουσιάζεται το σύνολο των πειραμάτων που πραγματοποιήθηκαν, τόσο σε επίπεδο προσομοίωσης, όσο και σε επίπεδο φυσικού ρομπότ, για την εξέταση και επαλήθευση της συμπεριφοράς των υποσυστημάτων της ρομποτικής πλατφόρμας Monstertruck, που στοχεύουν στην επίτευξη της αυτόνομης λειτουργίας της.
	\item Στο \textbf{Κεφάλαιο~\ref{Chapter6}} γίνεται μία έκθεση των συμπερασμάτων που προέκυψαν βάσει της ανάλυσης και των πειραμάτων που πραγματοποιήθηκαν για κάθε ένα από τα υποσυστήματα της ρομποτικής πλατφόρμας Monstertruck, ενώ, τελικά, αναφέρεται και ένα σύνολο ιδεών και προτάσεων για την βελτίωση της υπάρχουσας υλοποίησης, τόσο ως προς το υλικό, όσο και ως προς τους αλγορίθμους και το αντίστοιχο λογισμικό.
\end{itemize}
