% Chapter Template

\chapter{Συμπεράσματα και Μελλοντική Εργασία} % Main chapter title

\label{Chapter6} % Change X to a consecutive number; for referencing this chapter elsewhere, use \ref{ChapterX}

%----------------------------------------------------------------------------------------
%	SECTION 1: Conclusions
%----------------------------------------------------------------------------------------
\section{Συμπεράσματα}
Στην παρούσα διπλωματική εργασία παρουσιάστηκε η ανάπτυξη ενός αυτόνομου ρομποτικού οχήματος με  τετραδιεύθυνση και τετρακίνηση, τόσο ως προς το υλικό, αλλά όσο και ως προς το λογισμικό που απαιτήθηκε για την επίτευξη της αυτόνομης λειτουργίας αυτού. Η ρομποτική πλατφόρμα Monstertruck που αναπτύχθηκε αποτελεί ένα ρομποτικό όχημα, ικανό να εξερευνήσει, αυτόνομα, ένα άγνωστο περιβάλλον, να το χαρτογραφήσει και να πλοηγηθεί αποτελεσματικά μέσα σε αυτό.

\bigskip
Στο Kεφάλαιο~\ref{Chapter2} παρουσιάστηκε ο εξοπλισμός, μηχανολικός, ηλεκτρολογικός και ηλεκτρονικός, που εγκαταστάθηκε στην ρομποτική πλατφόρμα Monstertruck, όπως επίσης και η σκοπιμότητα του, για τη επίτευξη της παραπάνω λειτουργικότητας. Έπειτα, έγινε μία κινηματική ανάλυση για την εξαγωγή των σχέσεων μεταξύ των συστημάτων μετάδοσης κίνησης από τους κινητήρες και σερβοκινητήρες του οχήματος στους τροχούς, όσον αφορά την κίνηση ώθησης και στρέψης αυτών. Επίσης, παρουσιάστηκε η κινηματική ανάλυση του οχήματος, μέσω προσαρμογής του ιδανικού κινηματικού μοντέλου τετραδιεύθυνσης, όσον αφορά την κίνηση του στο επίπεδο, όπου παράχθηκε μία σχέση μεταξύ της επιθυμητής τροχιάς, με την αντίστοιχη απαιτούμενη κίνηση των τροχών. Βάσει των πειραμάτων που πραγματοποιήθηκαν για την αξιολόγηση του κινηματικού μοντέλου της ρομποτικής πλατφόρμας Monstertruck, συμπεραίνεται ότι λόγω του μη ιδανικού μηχανισμού στρέψης των τροχών, που έχει σαν αποτέλεσμα την ολισθαίνουσα κίνηση των τροχών, το κινηματικό μοντέλο παρουσιάζει μία μικρή απόκλιση από την ιδανική συμπεριφορά, όσον αφορά την κίνηση σε τροχιές αρνητικής τετραδιεύθυνσης. Aντιθέτως για την περίπτωση των τροχιών θετικής τετραδιεύθυνσης η απόκλιση από την ιδανική συμπεριφορά είναι αμελητέο έως μηδενική, λόγω της προσεγγιστικά παράλληλης στρέψης των τροχών του οχήματος, γεγονός που καθιστά την ρομποτική πλατφόρμα Monstertruck, καταλληλότερη για τροχιές αρνητικής τετραδιεύθυνσης, από ένα όχημα με ιδανικό μοντέλο τετραδιεύθυνσης.

\bigskip
Στο Κεφάλαιο~\ref{Chapter3} παρουσιάστηκαν οι αλγόριθμοι εκτίμησης κατάστασης, χαρτογράφησης και εντοπισμού θέσης, που χρησιμοποιήθηκαν στην ρομποτική πλατφόρμα Monstertruck, όπως επίσης και οι αλγόριθμοι που αναπτύχθηκαν και χρησιμοποιήθηκαν για την επίτευξη της αυτόνομης πλοήγησης αυτής σε ένα άγνωστο περιβάλλον.

\bigskip
Όσον αφορά την ταυτόχρονη χαρτοφράφηση και εντοπισμό θέσης, χρησιμοποιήθηκαν οι αλγόριθμοι CRSM-SLAM και Gmapping. Ο αλγόριθμος CRSM-SLAM αποδείχθηκε πολύ εύκολος στην χρήση και παραμετροποίηση, με πολύ μικρές απαιτήσεις σε υπολογιστικούς πόρους και απαιτούμενες πηγές δεδομένων. Συγκεκριμένα, ο αλγόριθμος απαιτεί μόνο σαρώσεις του περιβάλλοντος μέσω ενός σαρωτή λέιζερ, για την χαρτογράφηση και τον εντοπισμό θέσης, μέσω αντιστοίχισης των διαδοχικών σαρώσεων. Το γεγονός αυτό τον καθιστά κατάλληλο για κλειστούς χώρους και εύρωστο όσον αφορά φαινόμενα ολίσθησης, αλλά και σε περίπτωση που δεν υπάρχει άλλη πηγή πληροφορίας για τον εντοπισμό θέσης, αλλά παράλληλα ακατάλληλο για την περίπτωση ομοιόμορφου περιβάλλοντος διαστάσεων μεγαλύτερων της εμβέλειας του σαρωτή λέιζερ. Αντιθέτως, ο αλγόριθμος Gmapping αποτελεί μία αρκετά πιο απαιτητική λύση, όσον αφορά την παραμετροποίηση, τους υπολογιστικούς πόρους και τις πηγές δεδομένων, μιας και χρησιμοποιεί σαρώσεις του περιβάλλοντος αλλά και την εκτίμηση της κατάστασης του ρομπότ, βάσει ενός συστήματος οδομετρίας. Επομένως, παράγει μία πιο αξιόπιστη αναπαράσταση του περιβάλλοντος και εκτίμηση της κατάστασης του ρομπότ, ενώ παράλληλα επιτρέπει την χαρτογράφηση σε περιβάλλον, οι διαστάσεις του οποίου ξεπερνούν την εμβέλεια του λέιζερ. Παρόλα αυτά, η αξιοπιστία του αλγορίθμου Gmapping εξαρτάται σε μεγάλο βαθμό από την αξιοπιστία του συστήματος οδομετρίας και άρα είναι επιρρεπείς σε σφάλματα λόγω μη ιδανικού συστήματος οδομετρίας, αλλά και σε περίπτωση φαινομένων ολίσθησης. Για την αξιολόγηση της συμπεριφοράς των δύο αλγορίθμων χαρτογράφησης χρησιμοποιήθηκαν περιβάλλοντα συμβατά και με τους δύο αλγορίθμους, βάσει των οποίων κατασκευάστηκαν παραπλήσιοι χάρτες για την αναπαράσταση αυτών, που δεν ήταν δίχως ατέλειες, αλλά κρίθηκαν επαρκείς για τις απαιτήσεις του προβλήματος που εξετάζει η διπλωματικής εργασίας.

\bigskip
Για την αυτόνομη πλοήγηση της ρομποτικής πλατφόρμας Monstertruck παρουσιάστηκαν δύο συστήματα που επιλύουν το πρόβλημα της αυτόνομης πλοήγησης ενός οχήματος με κινηματικό μοντέλο τετραδιεύθυνσης. Το πρώτο σύστημα, παράγει ένα ολικό μονοπάτι, στον ελεύθερο χώρο, μεταξύ της αρχικής θέσης του ρομπότ και του δεδομένου στόχου και εφαρμόζει μία ακολουθία από στάδια επεξεργασίας για την παραμόρφωση του ολικού μονοπατιού σε ένα τοπικό ασφαλές και κινηματικά εφικτό μονοπάτι που μπορεί να ακολουθήσει το ρομπότ. Το δεύτερο σύστημα παράγει ένα αρχικό ολικό μονοπάτι, το οποίο έπειτα, ανακατασκευάζει συνεχώς, δυναμικά, όσο κινείται το ρομπότ. Για την διάσχιση του παραμορφωμένου ολικού μονοπατιού στην πρώτη περίπτωση, ή του ανακατασκευαζόμενου ολικού μονοπατιού στην δεύτερη περίπτωση, παρουσιάστηκε ένας αλγόριθμος διάσχισης μονοπατιού, βασισμένος σε ασαφή λογική και κατάλληλος για οχήματα με κινηματικό μοντέλο τετραδιεύθυνσης ή Ackermann. Βάσει των πειραμάτων που πραγματοποιήθηκαν στο Κεφάλαιο~\ref{Chapter5}, για την αξιολόγηση και σύγκριση των δύο συστημάτων αυτόνομης πλοήγησης, παρατηρήθηκε επιτυχής πλοήγηση και από τα δύο συστήματα με το δεύτερο να πετυχαίνει μικρότερο χρόνο πλοήγησης και μήκος τροχιάς για την επίτευξη ενός δεδομένου στόχου, αλλά με το δεύτερο να εκτελεί πιο ασφαλείς τροχιές μακρυά από τα εμπόδια. Επίσης, το σύστημα με δυναμική παραμόρφωση μονοπατιού παρουσίασε μεγαλύτερη ευρωστία, με την επίτευξη όλων των δυνατών στόχων και αποφυγή συγκρούσεων, αλλά αρκετά ασταθές τοπικό μονοπάτι σε περιπτώσεις που απαιτείται ελιγμός αναστροφής σε πολύ πυκνό από εμπόδια περιβάλλον. Αντιθέτως, το σύστημα με δυναμική ανακατασκευή μονοπατιού, παράγει, δυναμικά, ένα ντετερμινιστικό κινηματικά εφικτό μονοπάτι, αλλά σε πολύ χαμηλή συχνότητα, με αποτέλεσμα, η μέθοδος να μην μπορεί να χρησιμοποιηθεί για υψηλές ταχύτητες, ενώ παράλληλα η απουσία πρόσθετου μηχανισμού αποφυγής εμποδίων στο στάδιο, μεταξύ της κατασκευής μονοπατιού και της διάσχισης αυτού, με υψηλότερη συχνότητα από την κατασκευή μονοπατιού καθιστά το σύστημα αυτό επιρρεπές σε συγκρούσεις, ιδιαίτερα σε περιπτώσεις, που καθυστερεί η ανακατασκευή του ολικού μονοπατιού, λόγω απαίτησης πολύπλοκου ελιγμού σε περιβάλλον πυκνό με εμπόδια.

\bigskip
Όσον αφορά τον αλγόριθμο διάσχισης μονοπατιού με ασαφή λογική που αναπτύχθηκε στα πλαίσια της παρούσας εργασίας, παρατηρήθηκε ότι η επιθυμητή συμπεριφορά του αλγορίθμου επιτεύχθηκε σε μεγάλο βαθμό, με αυτόν να ακολουθεί αρκετά πιστά τα δεδομένα μονοπάτια, ενώ παράλληλα επαναφέρει το ρομπότ στο επιθυμητό μονοπάτι σε περίπτωση σημαντικής απόκλισης, όπως για παράδειγμα σε περίπτωση ολίσθησης, όπως παρουσιάστηκε στα αντίστοιχα πειράματα στην ενότητα \ref{ssec:path_tracking_experiments}. Το μόνο αρνητικό του αλγορίθμου είναι ότι δεν περιλαμβάνει κάποια μέθοδο αναγνώρισης περιπτώσεων σημαντικής απόκλισης από το δοσμένο μονοπάτι, γεγονός που θέτει σε κίνδυνο σύγκρουσης το ρομπότ, ιδιαίτερα σε περιπτώσεις χαμηλής συχνότητας ανανέωσης του μονοπατιού, ενώ παράλληλα δεν διαθέτει ούτε μηχανισμό εντοπισμού επικείμενης σύγκρουσης.

\bigskip
Τέλος, βάσει των παραπάνω, εξάγεται το συμπέρασμα ότι ο στόχος της παρούσας διπλωματικής, για την ανάπτυξη ενός αυτόνομου ρομποτικού οχήματος με κινηματικό μοντέλο τετραδιεύθυνσης και δυνατότητες χαρτογράφησης, εκτίμησης κατάστασης και αυτόνομης πλοήγησης σε άγνωστο περιβάλλον, επιτεύχθηκε σε σημαντικό βαθμό. Παρόλα αυτά, η υλοποίηση του συστήματος της ρομποτικής πλατφόρμας Monstertruck, αφήνει πολλές δυνατότητες βελτίωσης, αλλά ακόμα και επέκτασης αυτής. Στην ακόλουθη ενότητα παρουσιάζονται ένα σύνολο ιδεών και προτάσεων για βελτίωση της υπάρχουσας υλοποίησης αλλά και επέκταση αυτής με πρόσθετη λειτουργικότητα.

%----------------------------------------------------------------------------------------
%	SECTION 2: Future Work
%----------------------------------------------------------------------------------------
\section{Μελλοντική Εργασία}
Η ρομποτική πλατφόρμα Monstertruck που αναπτύχθηκε στα πλαίσια της παρούσας διπλωματικής εργασίας μπορεί να βελτιωθεί και να επεκταθεί βάσει των ακόλουθων προτάσεων, σχετικών με την μηχανολογική κατασκευή, τον ηλεκτρονικό εξοπλισμό και τον αλγοριθμικό σχεδιασμό:

\begin{itemize}
	\item Βελτίωση της μηχανολογικής κατασκευής του ρομποτικού οχήματος με στόχο την πλοήγηση σε περιβάλλοντα, με υψηλότερη δυσκολία προσπελασιμότητας, μέσω χρήσης μεγαλύτερων τροχών, μείωση του κέντρου βάρους του οχήματος και αντικατάσταση των πλαστικών τμημάτων του συστήματος μετάδοσης κίνησης, με μεταλλικά.
	\item Βελτίωση του κινηματικού μοντέλου του οχήματος, μέσω κατασκευής μηχανισμού στρέψης των τροχών που προσεγγίζει την ιδανική συνθήκη τετραδιεύθυνσης \ref{eq:4ws_condition}, ή μέσω κατασκευής μηχανισμού ανεξάρτητης στρέψης όλων των τροχών, κάτι που παράλληλα θα προσφέρει και δυνατότητα επιτόπου στροφής, μέσω στρέψης των τροχών κατά τρόπο που να έχει σαν αποτέλεσμα το κέντρο στιγμιαίας περιστροφής (ICR) του ρομπότ να βρίσκεται στο γεωμετρικό κέντρο του.
	\item Αύξηση της μέγιστης ταχύτητας του οχήματος, χρησιμοποιώντας έναν κινητήρα με υψηλότερη ταχύτητα περιστροφής, ή τροποποιώντας κατάλληλα το σύστημα τετρακίνησης, για μείωση του αντίστοιχου λόγου μετάδοσης.
	\item Προσθήκη κωδικοποιητών στους τροχούς του οχήματος για την βελτίωση της εκτίμησης της κατάστασης του ρομπότ, που παράγει το σύστημα οδομετρίας του.
	\item Προσθήκη αισθητήρα βάθους με στόχο τον εντοπισμό εμποδίων που δεν γίνονται αντιληπτά από τον σαρωτή λέιζερ, αλλά και γενικότερα εντοπισμό μη προσπελάσιμων τμημάτων του χώρου, όπως ράμπες μεγάλης κλίσης, σκάλες, γκρεμούς κλπ.
	\item Χρήση οπτικής οδομετρίας, μέσω κάμερας, για πιο εύρωστη εκτίμηση κατάστασης, αλλά και συμπλήρωση εκτίμησης πλήρους 6-DOF κατάστασης $[x,y,z,yaw,pitch,roll]$, με την προσθήκη εκτίμησης κατακόρυφης μετατόπισης.
	\item Προσθήκη λειτουργικότητας ρομποτικής όρασης για αναζήτηση σημείων ενδιαφέροντος, ανάλογα με την εφαρμογή.
	\item Αύξηση της υπολογιστικής ισχύος του ρομπότ, μέσω χρήσης ενός πιο ισχυρού υπολογιστή, ή ενός κατανεμημένου συστήματος υπολογιστών, ανάλογης ισχύος με τον υπολογιστή ODROID-XU4, όπου κάθε ένας θα είναι υπεύθυνος για ένα τμήμα της συνολικής λειτουργικότητας του ρομπότ, ενώ παράλληλα θα μπορούν να επικοινωνούν μεταξύ τους για τον συνδυασμό της παραγόμενης πληροφορίας. Αυτή η πρόταση κρίνεται απαραίτητη για είναι δυνατή η προσθήκη νέων αισθητήρων και λειτουργιών, καθώς, βάσει της υπάρχουσας λειτουργικότητας του ρομπότ και της υπάρχουσας υπολογιστικής ισχύος, δεν είναι δυνατή τέτοια επέκταση του συστήματος, χωρίς σοβαρές συνέπειες στην απόδοση του.
	\item Βελτίωση των δυνατοτήτων χαρτογράφησης και εντοπισμού θέσης με εγκατάσταση σαρωτή λέιζερ με μεγαλύτερη εμβέλεια, από τον υπάρχον αισθητήρα Hokuyo URG-04LX, ή και μεγαλύτερο οπτικό πεδίο, όπως για παράδειγμα ο αισθητήρας RPLIDAR που έχει εμβέλεια $6m$ και οπτικό πεδίο $360^\circ$.
	\item Επίσης, μπορούν να χρησιμοποιηθούν και αισθητήρες απόστασης (sonar/ir range sensors) για την μέτρηση των αποστάσεων στην νεκρή γωνία του υπάρχοντος σαρωτή λέιζερ, με αποτέλεσμα τον καλύτερο εντοπισμό εμποδίων πίσω από το ρομπότ, λειτουργία εξαιρετικής σημασίας, ειδικά κατά την εκκίνηση της λειτουργίας του που δεν γνωρίζει, εάν ο χώρος πίσω του είναι ελεύθερος ή κατειλημμένος και άρα δεν μπορεί να τον χρησιμοποιήσει για κάποιον ελιγμό, σε περίπτωση που ο χώρος μπροστά του είναι κατειλημμένος. 
	\item Βελτίωση της εκτίμησης κατάστασης του ρομπότ, μέσω ανάπτυξης ενός πιο εύρωστου συστήματος εκτίμησης κατάστασης για ρομποτικά οχήματα με τετραδιεύθυνση, βάσει Εκτεταμένων Φίλτρων Kalman, που συνδυάζουν πληροφορία από το σύστημα οδομετρίας του ρομπότ, από αισθητήρα IMU ή πυξίδα και από οπτική οδομετρία με κάμερα.
	\item Βελτίωση της συμπεριφοράς του συστήματος αυτόνομης πλοήγησης με δυναμική παραμόρφωση μονοπατιού, μέσω αντικατάστασης των έτοιμων υλοποιήσεων των αλγορίθμων ελαστικής ζώνης και κατασκευής μονοπατιών, που χρησιμοποιήθηκαν για την υλοποίηση του αλγορίθμου Reeds-Shepp Band. Συγκεκριμένα, προτείνεται η υλοποίηση του αλγορίθμου Bubble Band των \citeauthor{dpm} \cite{dpm}, όπως επίσης και η υλοποίηση ενός αλγορίθμου κατασκευής μονοπατιών Reeds-Shepp, με ντετερμινιστικό τρόπο και εξέταση της δυνατότητας παραλληλοποίησης των δύο. Επίσης, θα μπορούσε να ληφθεί υπόψιν και η μελέτη των \citeauthor{reeds_shepp_4ws} \cite{reeds_shepp_4ws} για την επέκταση του συνόλου των τύπων των μονοπατιών με μονοπάτια θετικής τετραδιεύθυνσης. Παράλληλα, το σύστημα αυτόνομης πλοήγησης θα βελτιωνόταν σημαντικά, και θα επέτρεπε την κίνηση σε αρκετά υψηλότερες ταχύτητες, μέσω αύξησης της συχνότητας λειτουργίας του αλγορίθμου διάσχισης μονοπατιού, ανεξάρτητα από την λειτουργία παραμόρφωσης μονοπατιού.
	\item Βελτίωση του αλγορίθμου διάσχισης μονοπατιού με προσθήκη λειτουργίας εντοπισμού επικείμενης σύγκρουσης, βάσει των πιο πρόσφατων μετρήσεων των αισθητήρων και αντίδραση μέσω μείωσης της ταχύτητας ή ακόμα και φρεναρίσματος, όπως παρουσιάζεται στην μελέτη των \citeauthor{reactive_fuzzy_ptc} \cite{reactive_fuzzy_ptc}.
	\item Εξέταση εναλλακτικών μεθόδων διάσχισης μονοπατιού μέσω προσαρμοστικού ελέγχου \cite{offroad_adaptive_control}, ή Machine Learning, όπως Reinforcement Learning \cite{rl_ptc}, ή υβριδικές προσεγγίσεις με Ασαφή Λογική και Νευρωνικά Δίκτυα \cite{neural_and_fuzzy_navigation} κλπ.
	\item Βελτίωση του συστήματος αυτόνομης πλοήγησης με δυναμική ανακατασκευή μονοπατιού, μέσω αντικατάστασης του αλγορίθμου διάσχισης μονοπατιού, με έναν αλγόριθμο που θα μπορεί να πραγματοποιεί τοπική αποφυγή εμποδίων, ενώ παράλληλα ακολουθεί, όσο πιο πιστά μπορεί, το ολικό μονοπάτι, λύνοντας έτσι το πρόβλημα κινδύνου σύγκρουσης, λόγω της χαμηλής συχνότητας ανακατασκευής του ολικού μονοπατιού, επιτρέποντας, παράλληλα, την δυνατότητα κίνησης με υψηλότερες ταχύτητες.
	\item Τέλος, 
\end{itemize}





